The old Elliptic Billiard did reveal to us a few more of its secrets. In a way it is like a cat: it never does what it is asked to do. But whatever it does is beautiful and surprising. Starting with the loci of triangular centers, you get a menu of elliptic, non-elliptic, and quasi-elliptic curves. Some are similar, others indentical to Billiard or Caustic. One is perfectly circular, another point-like.

Any combination of well-known invariants will produce another invariant. One surprise is that for $N=3$ the product of perimeter and angular momentum is manifested visibly as the constant ratio of Inradius to Circumradius. Equivalently, as constant orbit cosine sum, tangential cosine product, and tangential-to-orbit area ratios. Remarkably, all $N=3$ properties are also verified $\forall{N}$ (area ratios are only true for odd $N$). We leave the reader with a few questions:

\begin{itemize}
    \item For $N=3$, what determines whether a triangular center or vertex will produce an elliptic vs some other type of locus?
    \item Are there ellipsoidal (3d) counterparts to these invariants?
    \item Which invariants are still true for self-intersecting orbits?
    \item Are there non-Euclidean invariants?
\end{itemize}